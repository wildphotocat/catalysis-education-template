\documentclass[11pt, a4paper]{article}

\usepackage{newtxtext,newtxmath}

% ---- 1. GEOMETRY & LAYOUT ----
\usepackage[top=2.8cm, bottom=2cm, left=1.5cm, right=1.5cm, headheight=2cm]{geometry}
\usepackage{microtype}
\sloppy
\usepackage{lastpage}
\usepackage{caption}
\usepackage{tcolorbox}
\usepackage{amsmath}
\usepackage{siunitx}
\usepackage{multicol}
\setlength{\columnsep}{1.2em}

% ---- 2. FONTS ----
\usepackage{mathpazo} % Palatino (Body)
\usepackage[scaled=0.92]{helvet} % Helvetica (Headings)
\usepackage{courier} % Courier (Code)

% ---- 3. COLORS ----
\usepackage{xcolor}
\definecolor{catalysisblue}{RGB}{0, 51, 102}
\definecolor{accentgray}{RGB}{100, 100, 100}

% ---- 4. SECTIONS STYLING ----
\usepackage{titlesec}
\titleformat{\section}{\color{catalysisblue}\large\sffamily\bfseries}{\thesection}{1em}{}
\titleformat{\subsection}{\color{black}\normalsize\sffamily\bfseries}{\thesubsection}{1em}{}

% ---- 5. IMAGES ----
\usepackage{graphicx}
\usepackage[font={small,sf,bf}, labelfont={color=catalysisblue}]{caption}

% ---- 6. BIBLIOGRAPHY (ACS STYLE) ----
\usepackage[numbers,super,sort&compress]{natbib}
\bibliographystyle{unsrtnat}

% ---- 7. LINKS ----
\usepackage{hyperref}
\hypersetup{
	bookmarksopen=true,
	bookmarksnumbered=true,
	pdfstartview=FitH,
	colorlinks=true,
	linkcolor=catalysisblue,
	urlcolor=catalysisblue,
	citecolor=catalysisblue
}

% ---- 8. METADATA VARIABLES ----
\usepackage{fancyhdr}

\makeatletter % START: Allow use of @ variables

% Define variables
\newcommand{\articletype}[1]{\def\@articletype{#1}}
\newcommand{\received}[1]{\def\@received{#1}}
\newcommand{\revised}[1]{\def\@revised{#1}}
\newcommand{\affiliation}[1]{\def\@affiliation{#1}}
\newcommand{\fulladdress}[1]{\def\@fulladdress{#1}}
\newcommand{\correspondencename}[1]{\def\@correspondencename{#1}}
\newcommand{\correspondenceemail}[1]{\def\@correspondenceemail{#1}}

% ---- CUSTOM TITLE BLOCK ----
\renewcommand{\maketitle}{
	\thispagestyle{plain}
	\begin{center}
		% Article Type
		{\sffamily \bfseries \footnotesize \color{white} \colorbox{catalysisblue}{\,\MakeUppercase{\@articletype}\,} \par}
		\vspace{0.3cm}
		
		% Title
		{\LARGE \sffamily \bfseries \color{catalysisblue} \@title \par}
		\vspace{0.2cm}
		
		% Author
		{\Large \bfseries \@author \par}
		\vspace{0.1cm}
		
		% Affiliation
		{\small \itshape \@affiliation \par}
		{\small \itshape \@fulladdress \par}
		
		% Correspondence
		\vspace{0.1cm}
		{\small \sffamily \textbf{Correspondence to:} \@correspondencename \par}
		{\small \sffamily \textbf{Email:} \ttfamily \color{catalysisblue} \@correspondenceemail \par}
		
		\vspace{0.2cm}
		
		% Dates
		{\footnotesize \sffamily \color{accentgray} Received: \@received \quad | \quad Revised: \@revised \par}
		
		\vspace{0.3cm}
		
		% Bottom Rule
		\hrule height 0.5pt \color{accentgray}
		\vspace{0.3cm}
	\end{center}
}

\makeatother % END: Stop using @ variables

% Set default page style
\pagestyle{fancy}
\fancyhf{} % Clear all
\renewcommand{\headrulewidth}{0.5pt}
\renewcommand{\headrule}{\hbox to\headwidth{\color{accentgray}\leaders\hrule height \headrulewidth\hfill}}
\lhead{\sffamily \small \color{catalysisblue} Catalysis Education Newsletter}
\rhead{\sffamily \small Page \thepage}
\renewcommand{\footrulewidth}{0pt}

% =========================================
% TEMPLATE COPYRIGHT & LICENSE
% =========================================
% This LaTeX template is provided by:
% Catalysis Education Newsletter
% National Centre for Catalysis Research (NCCR) Alumni
% Indian Institute of Technology Madras
%
% Copyright (c) 2026 Professor Balasubramanian Viswanathan
%
% This work is licensed under the Creative Commons 
% Attribution-NonCommercial 4.0 International License (CC BY-NC 4.0).
%
% Under this license, users are permitted to copy, redistribute, 
% remix, transform, and build upon the material for non-commercial 
% purposes only, provided that appropriate credit is given to the 
% original author(s) and the source.
%
% To view a copy of this license, visit:
% http://creativecommons.org/licenses/by-nc/4.0/
%
% =========================================

% =========================================
% DOCUMENT CONTENT STARTS HERE
% =========================================

% FILL IN YOUR ARTICLE METADATA
\articletype{ARTICLE TYPE} % e.g., REVIEW, RESEARCH, TUTORIAL, PERSPECTIVE
\received{Month Day, Year}
\revised{Month Day, Year}

\title{Your Article Title Here}
\author{Author Name(s)}
\affiliation{Department or Research Group}
\fulladdress{Institution, City, Country}
\correspondencename{Corresponding Author Name}
\correspondenceemail{email@example.com}

\begin{document}
	
	\twocolumn[
	\maketitle
	
	\begin{quotation}
		\noindent \textbf{\sffamily \color{black}
			Write your abstract or key message here. This should be a concise summary of your article, typically 150-250 words. Explain the main topic, approach, and key findings or insights. This section will appear in a highlighted box at the beginning of your article.
		}
	\end{quotation}
	
	\vspace{0.3cm}]
	
	\section{Introduction}
	
	Begin your article here. Introduce the topic, provide context, and explain why this work is important for catalysis education. You can cite references like this\cite{example2024}.
	
	Use this template for educational articles, reviews, tutorials, perspectives, or teaching materials related to catalysis and surface science.
	
	\section{Main Content}
	
	Organize your content into logical sections. Each section should have a clear purpose and flow naturally from the previous one.
	
	\subsection{Subsections}
	
	You can use subsections to organize complex topics. Use them to break down your content into manageable parts.
	
	\subsection{Including Equations}
	
	You can include mathematical equations. For inline equations, use \$E = mc$^2$. For display equations:
	
	\begin{equation}
		\int_{0}^{\infty} e^{-x^2} dx = \frac{\sqrt{\pi}}{2}
	\end{equation}
	
	For important equations, you can highlight them:
	
	\begin{tcolorbox}[colback=blue!5!white, colframe=blue!60!black, boxrule=0.8pt, arc=3pt]
		\begin{equation}
			\Delta G = \Delta H - T\Delta S
		\end{equation}
	\end{tcolorbox}
	
	\subsection{Including Figures}
	
	Include figures to illustrate your points:
	
	\begin{figure}[h]
		\centering
		% \includegraphics[width=0.8\linewidth]{your_figure.pdf}
		\fbox{\parbox{0.8\linewidth}{\centering\vspace{2cm}[Place your figure here]\vspace{2cm}}}
		\caption{Write a descriptive caption for your figure. Explain what the figure shows and why it is important.}
		\label{fig:example}
	\end{figure}
	
	You can reference figures in the text like this: As shown in Figure~\ref{fig:example}, ...
	
	\subsection{Including Tables}
	
	Tables are useful for presenting data:
	
	\begin{table}[h]
		\centering
		\caption{Example table showing sample data}
		\label{tab:example}
		\begin{tabular}{lcc}
			\hline
			\textbf{Parameter} & \textbf{Value 1} & \textbf{Value 2} \\
			\hline
			Temperature (°C) & 25 & 100 \\
			Pressure (bar) & 1 & 10 \\
			Yield (\%) & 65 & 85 \\
			\hline
		\end{tabular}
	\end{table}
	
	\section{Discussion}
	
	Discuss the implications of your work. What are the key takeaways? How does this contribute to catalysis education?
	
	\section{Conclusion}
	
	Summarize the main points of your article. What should readers remember? What are the practical implications?
	
	\section*{Acknowledgments}
	
	If you used AI tools (such as Claude, ChatGPT, or others) in preparing your manuscript, please acknowledge them here. For example:
	
	\textit{The author acknowledges the use of [AI Tool Name] for [specific purpose, e.g., organizing notes, editing, literature synthesis].}
	
	You can also acknowledge funding sources, colleagues who provided feedback, or institutions that supported your work.
	
	\section*{Disclaimer}
	
	The views, opinions, and interpretations expressed in this article are solely those of the author(s) and do not necessarily reflect the official position or views of Catalysis Education Newsletter.
	
	% =========================================
	% BIBLIOGRAPHY
	% =========================================
	
	\begin{thebibliography}{99}
		
		\bibitem{example2024}
		Author, A.; Author, B.
		\href{https://doi.org/10.xxxx/xxxxx}{Article Title}.
		\textit{Journal Name}
		\textbf{Year}, \textit{Volume}, Pages.
		
		% Add more references as needed
		% Always include DOI links when available
		
	\end{thebibliography}
	
	\section*{Submission Information}
	
	\textbf{How to Submit:}
	\begin{itemize}
		\item Email your completed manuscript to: \texttt{catalysiseducation@gmail.com}
		\item Include all figure files separately (PDF or high-resolution images preferred)
		\item Provide complete citations with DOI links
		\item Clearly acknowledge any use of AI tools
	\end{itemize}
	
	\textbf{Questions?} Contact us at \texttt{catalysiseducation@gmail.com}
	
\end{document}